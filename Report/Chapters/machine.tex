Our experiment machine is an HP Pavilion Elite HPE workstation, and the system running on this machine is Ubuntu 16.04.1. The specs of this machine and the operating system will be stated below.

\subsection{CPU}

The processor on this machine is an Intel(R) Core(TM) i7 CPU 860 @ 2.80GHz with 4 cores. This CPU supports constant tsc feature, which makes cycle counting much more trivial. When running this project, we disabled 3 of the 4 cores, and made only a single core available for running our measurement benchmark. The specs summery for this core is listed below:

\begin{itemize}
    \item The frequency of this CPU should be 2.80GHz (read from model infomation). After a simple measurement, we found out that the frequency of the clock of this core is 2.793291 GHz, and the clock period of this core should be $$ \frac{1.000000}{2.793291 GHz} = 0.358001 ns $$
    \item This core has 64K on core L1 cache, which 32K of them are data cache, and the rest 32K are instruction cache
    \item This core has 25k on core unify L2 cache
    \item The L3 cache is 8M, and it is shared by all cores. Since we only enable single core when running, we can seem it as a private 8M L3 cache for this core.
\end{itemize}

\subsection{Memory}

This machine has totally 8G main memory.

\subsection{Buses}

\subsection{Disk}

\subsection{NIC}

\subsection{Operating System}
When running this project, the operating system running on the machine is normal Ubuntu 16.04.1. The system is running under 64 bit mode.
